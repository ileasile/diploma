\section{Модифицированная задача}

\subsection{Постановка задачи}
\par Пусть, как и прежде, $L$ -- непустое конечное множество, называемое \textbf{множеством меток}. Пусть $G(V, E)$, $G'(V', E')$ -- ориентированные графы. Граф $G'$, кроме того, связный. Будем называть граф $G$ архивным графом, граф $G'$ -- графом-паттерном, или шаблонным графом.

Введём отображения $l : V \to L$, $l' : V' \to L$, сопоставляющие вершинам архивного и шаблонного графов соответствующие метки. Никаких дополнительных требований на эти отображения мы уже не накладываем.

Пусть также $\mathcal{X}$ -- это непустое (возможно, бесконечное) множество \textbf{(множество характеристик)} с заданным на нём бинарным отношением $\rho:\mathcal{X} \times \mathcal{X} \to \{0, 1\}$

Мы вводим множество характеристик для того, чтобы, например, работать со взвешенными графами. Действительно, если положить 

\[\mathcal{X} = \mathbb{R}\], 
\[\rho(x, y) = \begin{cases}
1, & |x - y| < \varepsilon \\
0, & |x - y| \ge \varepsilon
\end{cases}, \varepsilon > 0\],

мы получим отношение $\rho$, заданное, как отношение <<примерного равенства>>. Его мы и будем использовать для сравнения весов дуг архивного графа и графа-паттерна.

Введём также <<помечающие отображения>> для дуг графа: $\chi : E \to \mathcal{X}$, $\chi' : E' \to \mathcal{X}$.

Несколько изменится и определение совпадения.

\begin{defn}
	Совпадением на графе $G$ будем называть частичный подграф $\widehat{G}(\widehat{V}, \widehat{E})$ графа $G$ такой, что:
	\begin{enumerate}
		\item Существует биективное отображение $m_{\widehat{G}}: V' \to \widehat{V}$.
		\item $\forall v' \in V': l'(v') = l(m_{\widehat{G}}(v'))$
		\item $\forall e' = (v^{\prime}_1, v^{\prime}_2) \in E': \rho(\chi'(e'), \chi((m_{\widehat{G}}(v^{\prime}_1), m_{\widehat{G}}(v^{\prime}_2)))) = 1$
		\item $\forall (v^{\prime}_1, v^{\prime}_2) \in E': (m_{\widehat{G}}(v^{\prime}_1), m_{\widehat{G}}(v^{\prime}_2)) \in E$
	\end{enumerate}
\end{defn} 

Пусть для вершины $v \in V$ существует некоторое совпадение  $\widehat{G}(\widehat{V}, \widehat{E})$ на графе $G$ такое, что $v \in \widehat{V}$. Тогда вершину $v$ будем называть подходящей паттерну $G'$ по совпадению $\widehat{G}$, иначе -- неподходящей. В случае, если вершина $v$ подходит по совпадению $\widehat{G}$, вершину $m_{\widehat{G}}^{-1}(v) \in  V'$ назовём соответствующей данной вершине $v$.

Через $\widetilde{V} \subseteq V$ обозначим множество всех подходящих паттерну $G'$ вершин. Наша задача и будет состоять в отыскании этого подмножества. Опишем используемый нами алгоритм.

Пусть $T = \emptyset \cup \mathcal{C}_{V'}^1$ -- все подмножества множества вершин графа-паттерна. Построим отображение $f_0 : V \to T$, заданное следующим:
\begin{equation}
v' \in f_0(v) \Leftrightarrow l(v) = l'(v').
\end{equation}

Нетрудно убедиться, что в силу инъективности отображения $l$, введённое отображение $f_0$ действительно имеет областью значений множество $T$.

Алгоритм будет строиться на изменении отображения $f_0$, поэтому для удобства нам потребуется ввести операцию над подобными отображениями. Пусть $f_1, f_2 : V \to 2^{V'}$. Обозначим $f_2 = Exclude(f_1, v_0 (\in V), v^{\prime}_0 (\in V'))$, если выполнено следующее:
\begin{enumerate}
	\item $\forall v \ne v_0 \in V: f_1(v) = f_2(v)$.
	\item $v^{\prime}_0 \notin f_2(v_0)$.
	\item $\{v^{\prime}_0\} \cup f_2(v_0) = f_1(v_0)$
\end{enumerate}

Ясно, что по отображению $f_1$ легко построить отображение $Exclude(f_1, v_0, v^{\prime}_0)$, просто исключая вершину $v^{\prime}_0$ из множества $f_1(v_0)$.
\subsection{Измененный алгоритм исключения по локальным условиям}

\subsection{Измененный алгоритм проверки циклов}