\section{Модифицированная задача}

\subsection{Постановка задачи}
\par Пусть, как и прежде, $L$ -- непустое конечное множество, называемое \textbf{множеством меток}. Пусть $G(V, E)$, $G'(V', E')$ -- ориентированные графы. Граф $G'$, кроме того, связный. Будем называть граф $G$ архивным графом, граф $G'$ -- графом-паттерном, или шаблонным графом.

Введём отображения $l : V \to L$, $l' : V' \to L$, сопоставляющие вершинам архивного и шаблонного графов соответствующие метки. Никаких дополнительных требований на эти отображения мы уже не накладываем.

Пусть также $\mathcal{X}$ -- это непустое (возможно, бесконечное) множество \textbf{(множество характеристик)} с заданным на нём бинарным отношением $\rho:\mathcal{X} \times \mathcal{X} \to \{0, 1\}$

Мы вводим множество характеристик для того, чтобы, например, работать со взвешенными графами. Действительно, если положить 

\[\mathcal{X} = \mathbb{R},\]
\[\rho(x, y) = \begin{cases}
1, & |x - y| < \varepsilon \\
0, & |x - y| \ge \varepsilon
\end{cases}, \varepsilon > 0,\]

мы получим отношение $\rho$, заданное, как отношение <<примерного равенства>>. Его мы и будем использовать для сравнения весов дуг архивного графа и графа-паттерна.

Введём также <<помечающие отображения>> для дуг графа: $\chi : E \to \mathcal{X}$, $\chi' : E' \to \mathcal{X}$.

Несколько изменится и определение совпадения.

\begin{defn}
	Совпадением на графе $G$ будем называть частичный подграф $\widehat{G}(\widehat{V}, \widehat{E})$ графа $G$ такой, что:
	\begin{enumerate}
		\item Существует биективное отображение $m_{\widehat{G}}: V' \to \widehat{V}$.
		\item $\forall v' \in V': l'(v') = l(m_{\widehat{G}}(v'))$
		\item $\forall e' = (v^{\prime}_1, v^{\prime}_2) \in E': \rho(\chi'(e'), \chi((m_{\widehat{G}}(v^{\prime}_1), m_{\widehat{G}}(v^{\prime}_2)))) = 1$
		\item $\forall (v^{\prime}_1, v^{\prime}_2) \in E': (m_{\widehat{G}}(v^{\prime}_1), m_{\widehat{G}}(v^{\prime}_2)) \in E$
	\end{enumerate}
\end{defn} 

Пусть для вершины $v \in V$ существует некоторое совпадение  $\widehat{G}(\widehat{V}, \widehat{E})$ на графе $G$ такое, что $v \in \widehat{V}$. Тогда вершину $v$ будем называть подходящей паттерну $G'$ по совпадению $\widehat{G}$, иначе -- неподходящей. В случае, если вершина $v$ подходит по совпадению $\widehat{G}$, вершину $m_{\widehat{G}}^{-1}(v) \in  V'$ назовём соответствующей данной вершине $v$.

Через $\widetilde{V} \subseteq V$ обозначим множество всех подходящих паттерну $G'$ вершин. Первая задача будет состоять в отыскании этого подмножества. В следующих трёх разделах описаны три используемых нами для её решения алгоритма.

Пусть $T = 2^{V}$ -- все подмножества множества вершин графа-паттерна. Построим отображение $f_0 : V' \to T$, заданное следующим:
\begin{equation}
v \in f_0(v') \Leftrightarrow l(v) = l'(v').
\end{equation}

В отличие от приведённого в предыдущей части работы, это отображение сопоставляет каждой вершине графа-паттерна вершину какое-то подмножество вершин архивного графа. Также это отображение является истинно многозначным, т.е. элементами его образа являются, в общем случае, не только 0- и 1-элементные подмножества множества $V$. 

В связи с изменением области значений и определения введённого отображения $f_0$ следует изменить и операцию $Exclude$. Пусть $f_1, f_2 : V' \to 2^{V}$. Обозначим $f_2 = Exclude(f_1, v_0 (\in V), v^{\prime}_0 (\in V'))$, если выполнено следующее:
\begin{enumerate}
	\item $\forall v' \ne v^{\prime}_0 \in V': f_1(v') = f_2(v')$.
	\item $v_0 \notin f_2(v^{\prime}_0)$.
	\item $\{v_0\} \cup f_2(v^{\prime}_0) = f_1(v^{\prime}_0)$
\end{enumerate}

\subsection{Измененный алгоритм исключения по локальным условиям}

Ниже приведён изменённый алгоритм исключения по локальным условиям. Он принимает те же самые графы, но входное и выходное отображение имеют другие сигнатуры. Общая идея алгоритма сохраняется.

\begin{algorithm}[H]
	\Large
	\KwIn{графы $G(V, E)$, $G'(V', E')$, отображение $f_0 : V' \to T$, число итераций $N$}
	\KwOut{измененённое отображение $f_K$}
	\Begin(MLCCE){
		$K := 0$
		
		$F_0 := f_0$
		
		\For{$i = 1, 2, .., N$}{
			\For{$(q_0, q) \in E'$}
			{
				\For{$v_0 \in f(q_0)$}{
					$flag := False$
					
					\For{$v \in \Gamma(v_0)$}{
						\If{$v \in f_K(q)$}{
							\If{$\rho(\chi(v_0, v), \chi'(q_0, q)) = 1$}{
								$flag := True$
							}
						}
					}
				
					\If{$flag = False$}{
						$f_{K+1} := Exclude(f_{K}, v, v')$
						
						$K := K + 1$
					}
				}
			}
			
			$F_i := f_K$
		}
		\Return{$f_N$}
	}
	
	\caption{Измененный алгоритм исключения по локальным условиям}
	\label{alg:MLCCE}
\end{algorithm}

Алгоритм $MLCCE$ полностью берёт идею алгоритма, изложенного выше, а потому его корректность прямо вытекает из корректности вышеизложенного алгоритма. Мы лишь добавили пару циклов для того, чтобы обрабатывать многозначность.

Сложность приведённого алгоритма, очевидно, выше сложности алгоритма $LCCE$. Оценим её на одной итерации:

\[T = \sum\limits_{(q_0, q) \in E'}  |f_i(q_0)| = \sum\limits_{q_0 \in V'}  |\Gamma(q_0)|\cdot|f_i(q_0)| \le \]

В худшем случае $f(q_0) = |V|$, имеем:

\[ \le |E'|\cdot|V| \]

Худший случай, однако, реализуется, только если задача поиска паттерна вырождается в задачу изоморфизма. Для реальных задач алгоритм $MLCCE$ работает достаточно быстро, за несколько итераций уменьшая число кандидатов на несколько порядков.

\subsection{Измененный алгоритм проверки контуров}

Для корректного исключения кандидатов на графах с контурами также был модифицирован алгоритм проверки контуров. Его изменённая версия приведена ниже.

\begin{algorithm}[H]
	\Large
	\KwIn{графы $G(V, E)$, $G'(V', E')$, отображение $f_0 : V' \to T$}
	\KwOut{измененённое отображение $f_N$}
	\Begin(CCE){
		$K := 0$
		
		\ForEach{$\mathcal{C}_0 \in \mathcal{K}_0$}{
			Пусть $(v_0^{\prime}, v_1^{\prime})$ -- первая дуга контура $\mathcal{C}_0$.
			
			\For{$v_0 \in f(v_0^{\prime})$}{
				$\mathcal{A} := \{v_0\}$
				
				\For{$s = 1, 2, ..., |\mathcal{C}_0|$}{
					Пусть $(q_0, q_1)$ -- $s$-я дуга контура $\mathcal{C}_0$.
					
					$\mathcal{B} := \emptyset$
					
					\ForEach{$v_0 \in \mathcal{A}$}{
						\ForEach{$v_1 \in f_K(q_1)$}{
							$flag = False$
							
							\If{$v \in \Gamma(v_0)$}{
								\If{$s \ne |\mathcal{C}_0| \lor v_0 = v_1$}{
									\If{$\rho(\chi(v_0, v_1), \chi'(q_0, q_1)) = 1$}{
										$flag := True$
									}
								}
							}
						
							\If{$flag = True$}{
								$\mathcal{B} := \mathcal{B} \cup \{v_1\}$
							}
						}
					}
					
					$\mathcal{A} := \mathcal{B}$ 
				}
			
				\If{$v_0 \notin \mathcal{A}$}{					
					$f_{K+1} := Exclude(f_{K}, v_0, v_0^{\prime})$
						
					$K := K + 1$
				}
			}
		}
		\Return{$f_N$}
	}
	
	\caption{Изменённый алгоритм проверки контуров}
	\label{alg:MCCE}
\end{algorithm}


\subsection{Общий алгоритм исключения кандидатов}

Общий алгоритм без каких-либо изменений переносится с уже введённого выше (алгоритм \ref{alg:EE}). Разница лишь в том, что передаваемое ему отображение формируется так, как указано выше в этом разделе.

Таким образом, построен алгоритм, решающий гораздо более общую задачу. Однако полученное после его работы отображение $f_N$ не может использоваться непосредственно -- необходимо ещё выделить совпадения на графе $G$. Соответствующему алгоритму посвящён следующий раздел.

\subsection{Рекурсивный алгоритм выделения совпадения}

Для восстановления совпадений по полученному отображению был разработан следующий метод:

\begin{enumerate}
	\item Преобразовать граф-паттерн $G'$ к особому виду, снабдив его дополнительной информацией, получив таким образом граф $G_0$.
	\item Положить $i := 0$.
	\item Пока граф $G_i$ имеет более одной вершины:
	\begin{enumerate}
		\item Выполнить на графе $G_i$ обход в ширину с окраской.
		\item Выделить подграфы графа $G_i$ с вершинами одного цвета.
		\item На каждом выделенном подграфе $G_i^j(V_i^j, E_i^j)$ построить все возможные соответствия между его вершинами и вершинами архивного графа.
		\item Сформировать новый граф $G_{i+1}$, взяв в качестве вершин подграфы $G_i^j$, а в качестве дуг -- дуги, вершины которых принадлежат сразу нескольким подграфам.
		\item Положить $i := i + 1$.
	\end{enumerate}
\end{enumerate}

Рассмотрим подробнее каждый из этих пунктов.

Преобразование графа $G'$ выполняется следующим образом: каждая его вершина $v'$ получает как характеристику список соответствий $\xi(v')$. Каждый список соответствий состоит из кортежей, элементами которых являются пары вида $(q_0, v_0)$, означающих, что вершине $q_0$ графа-паттерна соответствует вершина $v_0$ архивного графа. Таким образом,

\[\forall v' \in V': \xi(v') = [((v', v_1), ..., (v', v_L))], v_1, ..., v_L \in f(v_L).\]

