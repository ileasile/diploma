\subsection{Решение задачи для бесконтурного случая}
\par Мы ограничимся решением задачи в случае, когда цветной граф G не содержит контуров. Введём на множестве $X$ вершин этого графа отношение $\preceq$: $x \preceq y$ тогда и только тогда, когда существует путь из вершины $x$ в вершину $y$, либо $x = y$. Известно, что введённое отношение на множестве вершин бесконтурного графа является отношением частичного порядка. Какое-либо упорядочение вершин $(x_1, x_2, ..., x_{|X|})$, для которого $\forall i, j \quad i < j \Rightarrow \neg (x_j \preceq x_i)$, назовём \textbf{топологической сортировкой} графа $G$. Известно, что для бесконтурного графа существует по крайней мере одна топологическая сортировка его вершин. Далее будем предполагать, что $(x_1, x_2, ..., x_{|X|})$ -- некоторая топологическая сортировка вершин графа $G$.
\par Напомним, что $t$ и $s$ -- начальная и конечная вершины рассматриваемого нами пути. Найдём $a, b \in \{1, 2, ..., |X|\}$ такие, что $x_a = s, x_b = t$. Если $a > b$, задача не имеет решения, поскольку не существует пути из $x_a$ в $x_b$ даже в обычном смысле. Далее будем считать, что $a \leq b$.
\par Введём в рассмотрение для каждой вершины $x_n, a \leq n \leq b$ матрицу вероятностей $P^{(n)} \in \mathbb{M}_{M \times M}$ такую, что $P^{(n)}_{i,j}$ - это вероятность оказаться в вершине $t$ в цвете $j$, начав движение из вершины $x_{n}$ в цвете $i$, и выбрав оптимальную стратегию движения. Непосредственно из определения следует, что $P^{(n)} = E$, где $E$ - единичная матрица.
\par Для нахождения остальных матриц $P^{(n)}$ мы будем использовать метод полной индукции, переходя от $(n + 1), (n + 2), ..., b$ к $n$. Пусть путешественник находится в вершине $x_n$ в цвете $i$. Тогда он может перейти по любой из дуг, допускающих этот цвет, то есть по любой из дуг множества $Inc_i(x_n)$. Пусть $\hat{X_i}(x_n)$ -- множество концов этих дуг. Путешественник не должен рассматривать в качестве кандидатов на переход те вершины $x_k \in \hat{X_i}(x_n)$, для которых $k > b$, поскольку после такого перехода вероятность оказаться в вершине $t = x_b$ окажется равной нулю. Заметим, что обязательно $k > n$, поскольку вершины графа топологически упорядочены. Поэтому множество вершин-кандидатов можно описать следующим образом: $X_i(x_n) = \{x_k \in \hat{X_i}(x_n) | n < k \leq b\}$. Если это множество пусто, то вероятность попасть в вершину $t$, находясь в цвете $i$, равна нулю, т.е. $\forall j \: P^{(n)}_{i,j} = 0$. Далее будем считать, что $X_i(x_n) \ne \emptyset$.
\par Пусть путешественник выбрал для перехода вершину $x_k \in X_i(x_n)$. Тогда матрица вероятностей для вершины $x_n$ при условии перехода в вершину $x_k$ будет равна $P^{(x_n, x_k)} = P \cdot P^{(k)}$. Это следует непосредственно из формулы полной вероятности.
\par Теперь необходимо выяснить, какую именно из вершин множества $X_i(x_n)$ должен выбрать путешественник. Мы договорились, что нашей целью является максимизация вероятности достигнуть конечной вершины $t$, при этом не имеет значения цвет, в котором эта вершина будет достигнута. Поэтому вершину для перехода необходимо выбирать так: 
\[v_i = \argmaxA_{x_k \in X_i(x_n)} \sum_{l = 1}^{M} P^{(x_n, x_k)}_{il}.\]
Таким образом, стратегия при нахождении в $i$-м цвете в вершине $x_n$ будет заключаться в выборе для перехода вершины $v_i$, а $i$-я строка искомой матрицы $P^{(n)}$ будет равна $i$-й строке матрицы $P^{(x_n, v_i)}$. Более формально, 

\[\forall j \: P^{(n)}_{i, j} = P^{(x_n, v_i)}_{i, j}, v_i = \argmaxA_{x_k \in X_i(x_n)} \sum_{l = 1}^{M} P^{(x_n, x_k)}_{il}.\]

\par Для решения задачи предлагается использовать следующий алгоритм:
\begin{enumerate}
	\item $\forall n > b: P^{(n)} := 0$.
	\item $P^{(b)} := E$.
	\item Для $n = b - 1, b - 2, ..., a$:
	\begin{enumerate}
		\item Инициализируем $P^{(n)} := 0$.
		\item Для всех $i = 1..M$:
		\begin{enumerate}
			\item Определим множество концов дуг, допускающих цвет $i$ -- $X_i(x_n)$.
			\item Если $X_i(x_n) \ne \emptyset$:
			\begin{enumerate}
				\item Для каждой вершины множества $x_k \in X_i(x_n)$ вычислим \mbox{$i$-ю} строку матрицы $P^{(x_n, x_k)}$ как указано выше, а также вероятность достижимости как сумму элементов этой строки: 
				\[p^{(x_n, x_k, i)} = \sum_{l = 1}^{M} P^{(x_n, x_k)}_{il}.\]
				\item Найдём вершину $v_{i, n} = \argmaxA_{x_k \in X_i(x_n)} p^{(x_n, x_k, i)}$.
				\item Присвоим: $\forall j = 1..M \! \!: P^{(n)}_{i, j} = P^{(x_n, v_{i, n})}_{i, j}$
			\end{enumerate}
			\item Если $X_i(x_n) = \emptyset$, установим $v_{i, n} = \bigotimes$. Это будет означать, что находясь в цвете $i$, из вершины $x_n$ добраться до конечной вершины невозможно. 
		\end{enumerate}
	\end{enumerate} 
	\item Возвращаем $P^{(a)}$ как искомую матрицу вероятностей для вершины $s$, а также набор вершин $(v_{i, n})_{a \le n < b, i \in C}$ как искомую стратегию.
\end{enumerate}

\par Очевидно, что алгоритм всегда заканчивается. Корректность алгоритма вытекает из описания, приведённого выше.
\par Определим теперь временную сложность алгоритма.
\par Первый и второй шаги алгоритма отрабатывают за время $O(M^2)$. На третьем шаге наиболее трудным является вычисление строк матрицы вероятностей. Вычисление одной такой строки занимает время $O(M^2)$. Количество таких вычислений: 

\[ S = \sum\limits_{n = a}^{b} \sum\limits_{i = 1}^{M} |X_i(x_n)| \le (b - a + 1) \cdot M \cdot \max\limits_{n, i} |X_i(x_n)|\]. 

Поскольку $(b - a + 1) \le |X|$, $\max\limits_{n, i} |X_i(x_n)| \le |X|$, то $S \le M |X|^2$. Поэтому общая сложность алгоритма составит $O(M^3 |X|^2)$.

\begin{comment}
\begin{lemma}
	В строках 3-9 функции $GetProbabilities$ вычисляются корректные вероятности достижимости для всех вершин, следующих за t в порядке топологической сортировки и для самой вершины t.
\end{lemma}
\begin{proof}
	Действительно, в строках 3-7, всем вершинам, начиная с последней вершины в порядке топологической сортировки и до тех пор, пока не встретится вершина t, назначаются нулевые вероятности достижимости вне зависимости от начального цвета, поскольку из всех этих вершин вершина t не достижима даже в обычном смысле.
	
	В строках 7-8 все вероятности достижимости (вне зависимости от начального цвета) для вершины t устанавливаются равными 1. Действительно, вне зависимости от начального цвета существует путь нулевой длины из вершины t в себя, и указанное ограничение на достижимость не влияет на его существование.
\end{proof}

\begin{lemma}
	На каждой итерации цикла в строке 11 (конечные) вероятности достижимости для вершины $x$ вычисляются верно, если правильно вычислены вероятности достижимости для всех вершин, следующих за $x$ в порядке топологической сортировки.
\end{lemma}
\begin{proof}
	Пусть путешественник находится в вершине x и при этом установлен цвет c. Тогда из вершины x он может перейти по любой из дуг множества $Inc_c(x)$, и только по этим дугам. Путешественник волен выбрать любую из этих дуг, его цель -- максимизировать свои шансы достигнуть вершины t. Пусть $Inc_c(x) = \{u_1 = (x,y_1), u_2 = (x,y_2), ..., u_n = (x,y_n)\}$. Тогда стратегия путешественника заключается в том, что он выбирает дугу $u_i$ с вероятностью $q_i$, причём $\sum\limits_{i=1}^{n} q_i = 1$. 
	
	В строках 16-21 для каждой из дуг $u_i$ вычисляется вероятность $pr$ достигнуть из неё вершины t в случае перехода по этой дуге. При этом используется формула полной вероятности: $pr = P(E) = \sum\limits_{cy \in C} P(H_{cy}) P(E | H_{cy})$. В качестве гипотез берутся события $H_{cy}$ -- "перейти по данной дуге и оказаться в цвете cy". Вероятности этих гипотез обсуждались выше. Условные вероятности $P(E | H_{cy})$ равны $prob[y][cy]$ и вычислены верно по предположению леммы. Таким образом, каждая из вероятностей $pr_i$ в строках 16-21 вычисляется верно.   
	
	Учитывая вышесказанное и снова пользуясь формулой полной вероятности, имеем:
	\begin{equation*}
	prob[x][c] = \sum\limits_{i=1}^{n} q_i\cdot pr_i \leq \sum\limits_{i=1}^{n} q_i \cdot \max\limits_{j = 1}^{n} pr_j = \max\limits_{j = 1}^{n} pr_j \cdot \sum\limits_{i=1}^{n} q_i = \max\limits_{j = 1}^{n} pr_j
	\end{equation*}
	Выберем произвольное $k \in \argmax\limits_{j=1}^{n} pr_j$. Пусть 
	\begin{equation*}
	q_i^* = \begin{cases} 
	1 & i = k \\
	0 & i \ne k \\
	\end{cases}
	\end{equation*}
	
	Для таких $q_i^*$ выполняется $\sum\limits_{i=1}^{n} q_i^* = 1$. Более того, беря такие $q_i^*$ в качестве стратегии, имеем:
	\begin{equation*}
	prob[x][c] = \sum\limits_{i=1}^{n} q_i^*\cdot pr_i = q_k^*\cdot pr_k = pr_k = \max\limits_{j = 1}^{n} pr_j
	\end{equation*}
	Таким образом, при данном выборе стратегии движения путешественник добивается максимальной вероятности достижимости вершины t. В строках 14-24 действительно вычисляется максимум по всем вероятностям $pr_k = \max\limits_{j = 1}^{n} pr_j$ и сохраняется в переменную $prob[x][c]$, а также в переменную $path[x][c]$ сохраняется соответствующая оптимальной стратегии вершина $y_k$, в которую путешественник должен будет перейти, находясь в вершине x в цвете c.
\end{proof}

\begin{theorem}
	Функция $GetProbabilities$ возвращает корректные вероятности достижимости и массив оптимальных путей для вершины $s$ и всех следующих за ней в порядке топологической сортировки вершин.
\end{theorem}
\begin{proof}
	Утверждение теоремы для вершины t и следующих за ней в порядке топологической сортировки совпадает с утверждением первой леммы. Пусть теперь V -- множество всех вершин, следующих за s в порядке топологической сортировки или совпадающих с ней, но строго предшествующих вершине t. Выберем из этого множества вершину с наибольшим номером в правильной нумерации. Тогда по второй лемме для неё выполнено утверждение теоремы. Продолжая извлекать из множества V вершины и, пользуясь второй леммой, убеждаться, что для них выполняется утверждение теоремы, мы наконец дойдём до вершины s, тем самым доказав теорему.
\end{proof}

\end{comment}