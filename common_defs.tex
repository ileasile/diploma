\section{Основные понятия}
Ниже приведены основные понятия и утверждения, необходимые для дальнейшего изложения \cite{berzh}.

\begin{defn}
	Ориентированным графом (в дальнейшем -- просто \textbf{графом}) будем называть двойку $G (V, E)$, где:
	\begin{itemize}
		\item $V, |V|>0$ - множество вершин графа;
		\item $E \subset V \times V$ - множество дуг графа;
	\end{itemize}
\end{defn}
\begin{defn}
	Граф $G(V, E)$ будем называть конечным, если множества $V$ и $E$ конечны.
\end{defn}
\begin{defn}
	Тройку $G(V, E, \omega)$ будем называть взвешенным графом, если $G'(V, E)$ -- граф, а $\omega : E \to \mathbb{R}$ -- весовая функция.
\end{defn}
\begin{defn}
	Будем говорить, что дуга $e = (u, v)$ исходит из вершины $u$ и заходит в вершину $v$. Вершину $u$ будем называть началом, а $v$ -- концом дуги.
\end{defn} 
\begin{defn}
	Будем говорить, что вершина $u$ смежна с вершиной $v$, если существует дуга $e = (u,v) \in E$. Обозначим множество вершин, смежных с $u$, через $\Gamma(u)$. Множество $Inc(u) = \{(u, v) \in E | v \in V \}$ назовём множеством инцидентных $u$ дуг.
\end{defn} 

Введём понятия пути и цепи.

\begin{defn}
	Путём называется последовательность дуг $\mu = (e_i = (u_i, u_{i+1}))_{i = 0}^{m-1}$. Число $m$ назовём длиной пути и обозначим через $|\mu|$. Будем говорить, что $\mu.s = u_0$ -- это начало пути $\mu$, а $\mu.t = u_m$ -- это конец пути $\mu$. Для удобства будем считать, что на графе $G$ существует $|V|$ путей нулевой длины, каждый из которых представляет собой пустую последовательность дуг, а его начало и одновременно конец суть некоторая вершина графа G. Путь ненулевой длины $\mu$ называется контуром, если его начало совпадает с концом. Если граф не содержит контуров, он называется бесконтурным.
\end{defn} 

Также приведём определение цепи.

\begin{defn}
	Цепью на графе $G$ называется последовательность дуг $\lambda = (e_i)_{i = 0}^{m-1}$ такая, что для любого $i \in [0;m-2]_\mathbb{N}$ дуги $e_i$ и $e_{i+1}$ имеют ровно одну общую вершину (но необязательно начало дуги $e_i$ совпадает с концом $e_{i+1}$). Число $m$ назовём длиной цепи и обозначим через $|\lambda|$. Будем говорить, что вершина дуги $e_0$, не инцидентная дуге $e_q$ -- это начало цепи $\lambda$, а вершина дуги $e_{m-1}$, не инцидентная дуге $e_{m-2}$ -- это конец цепи $\lambda$. Цепь ненулевой длины $\lambda$ называется циклом, если её начало совпадает с концом.
\end{defn} 

Необходимо также ввести некоторые определения, специфичные для данной работы.

\begin{defn}
	Расстоянием $d(u, v)$ между вершинами графа $u$ и $v$ называется длина наименьшего пути, началом которого является вершина $u$, а концом - вершина $v$, или наоборот. Если ни одного такого пути не существует, то расстояние между вершинами полагается равным бесконечности.
\end{defn}

\begin{defn}
	Неориентированным расстоянием $dn(u, v)$ между вершинами графа $u$ и $v$ называется длина наименьшей цепи, началом которой является вершина $u$, а концом - вершина $v$. Если ни одной такой цепи не существует, то неориентированное расстояние между вершинами полагается равным бесконечности.
\end{defn}

\begin{defn}
	Диаметром $diam(G)$ связного графа $G$ называется максимальное из неориентированных расстояний между его вершинами.
\end{defn}

Заметим, что в связном конечном графе неориентированные расстояния между любыми двумя вершинами всегда конечны, поэтому и диаметр такого графа также будет конечен. В данной работе мы будем рассматривать лишь связные и конечные графы, поэтому все они будут иметь конечный диаметр.

Нам также понадобятся определения, связанные с подграфами и частичными графами.

\begin{defn}
	 Граф $G'(V', E')$ называется частичным графом графа $G(V, E)$, если $V' = V, E' \subset E$.
\end{defn}

\begin{defn}
	Граф $G'(V', E')$ называется подграфом графа $G(V, E)$, если $V' \subset V, E' = \{(v, u) \in E | v \in V', u \in V'\}$.
\end{defn}

\begin{defn}
	Граф $G'(V', E')$ называется частичным подграфом графа $G(V, E)$, если $G'$ является частичным графом некоторого подграфа графа $G$.
\end{defn}