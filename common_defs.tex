\section{Основные сведения}
Дадим основные определения, которые будут использоваться в этой работе.

\begin{defn}
	\textbf{Ориентированным графом} (в дальнейшем -- просто \textbf{графом}) будем называть тройку $G (X, U, f)$, где:
	\begin{itemize}
		\item $X, |X|>0$ - множество вершин графа;
		\item $U$ - множество дуг графа;
		\item $f : U \to X\times X $ - отображение, сопоставляющее каждой дуге её начало и конец;
	\end{itemize}
\end{defn}
\begin{defn}
	Граф $G(X, U, f)$ будем называть \textbf{конечным}, если множества $X$ и $U$ конечны.
\end{defn}
\begin{defn}
	Четвёрку $G(X, U, f, \omega)$ будем называть \textbf{взвешенным графом}, если $G'(X, U, f)$ -- граф, а $\omega : U \to \mathbb{Z}$ -- весовая функция.
\end{defn}
\begin{defn}
	Будем говорить, что дуга $u$ \textbf{исходит} из вершины $x$ и \textbf{заходит} в вершину $y$, если $f(u) = (x,y)$. Обозначим $f_+(u)=x$, $f_-(u)=y$.
\end{defn} 
\begin{defn}
	Будем говорить, что вершина $y$ \textbf{смежна} с вершиной $x$, если $\exists u : f(u) = (x,y)$. Обозначим множество вершин, смежных с $x$, через $Adj(x)$. Множество $Inc(x) = \{u \in U | f_+(u) = x  \}$ назовём множеством \textbf{инцидентных} с $x$ дуг.
\end{defn} 
\begin{defn}
	\textbf{Путём} называется последовательность дуг $\widehat{u} = (u_i)_{i = 1}^m$ такая, что $\forall i\in [1;m-1]_\mathbb{N} \quad f_-(u_i) =f_+(u_{i+1})$. Число $m$ назовём \textbf{длиной пути} и обозначим через $|\widehat{u}|$. Будем говорить, что $\widehat{u}.s = f_+(u_1)$ -- это \textbf{начало} пути $\widehat{u}$, а $\widehat{u}.t = f_-(u_m)$ -- это \textbf{конец} пути $\widehat{u}$. Для удобства будем считать, что на графе $G$ существует $|X|$ путей нулевой длины, каждый из которых представляет собой пустую последовательность дуг, а его начало и одновременно конец суть некоторая вершина графа G. Путь ненулевой длины $\widehat{u}$ называется \textbf{контуром}, если его начало совпадает с концом. Если граф не содержит циклов, он называется \textbf{бесконтурным}.
\end{defn} 
\begin{defn}
	\textbf{Весом пути} взвешенного графа назовём сумму весов его рёбер: 
	\begin{equation*}
	\omega(\widehat{u}) = \sum\limits_{k=1}^{|\widehat{u}|} \omega(u_k).
	\end{equation*}
\end{defn} 

\begin{defn}
	Тройку $G( G'(X, U, f [, \omega]), C, \{U_c\}_{c \in C} )$ будем называть \textbf{цветным графом}, если $G'$ -- конечный граф, $C$ -- конечное непустое множество, $\{U_c\}_{c \in C}$ -- семейство множеств таких, что $U_c \subset U$ и $U = \bigcup\limits_{c \in C} U_c$. Заметим, что это семейство не обязательно является разбиением множества $U$. Множество $C$ будем называть множеством цветов, а каждое из множеств $U_c$ - множеством дуг, допускающих цвет $c$. Обозначим через $Inc_c(x)$ множество дуг, допускающих цвет $c$ и инцидентных вершине $x$.
\end{defn}

Заметим, что исходя из определения каждая дуга допускает по крайней мере один цвет.