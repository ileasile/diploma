\section{Введение}
Теория графов - относительно молодой и быстроразвивающийся раздел современной математики. В ней присутствует достаточно много серьёзных теоретических проблем, таких как, например, гипотеза Харари\cite{harari} о том, что если граф имеет более трёх рёбер, то его можно однозначно восстановить по подграфам, полученным удалением единственного ребра. Но вместе с этим теория графов успешно применяется для решения прикладных задач, возникающих в теории компьютерных сетей, машинном обучении, при проектировании и эксплуатации транспортных систем, в теории игр и т.д.

Одной из классических задач теории графов является задача об изоморфизме графов. В ней даны два графа $G$ и $G'$, для которых необходимо построить два биективных отображения $\delta'$ и $\gamma$ таким образом, чтобы они переводили один граф в другой. Понятно, что такие отображения построить можно не всегда. Долгое время для этой задачи не могли найти хорошего алгоритма, и лучшей оставалась оценка $T(n) = \exp(O(\sqrt{n \log n}))$, где $n$ - число вершин, но в 2015 году Ласло Бабай опубликовал статью \cite{izom}, в которой показывается оценка $T(n) = \exp((\log n)^{O(1)})$, которая очень близка к полиномиальной. Тем не менее, приведённый в этой статье алгоритм всё ещё имеет слишком большую вычислительную сложность для того, чтобы применяться на практике для больших графов. Тем более, задача зачастую стоит немного по-другому: в графе $G$ надо найти частичный подграф \cite{berzh}, изоморфный графу $G'$, что сразу усложняет любой существующий -- переборный или более продвинутый -- алгоритм.

В приложениях, однако, часто можно ослабить условие задачи изоморфизма, определенным образом пометив вершины или дуги графов $G$ и $G'$, то есть сопоставив каждой вершине этих графов какой-то элемент множества $L$, а каждой дуге -- элемент множества $M$. В этом случае на отображения $\delta$ и $\gamma$ накладываются достаточно жесткие условия: $\delta$ может перевести $x$ в $x'$, а $\gamma$ -- $u$ в $u'$, только если вершины $x$ и $x'$ и дуги $u$ и $u'$ помечены одинаково. В случае, если множества меток $L$ и $M$ достаточно велики, количество перебираемых для отображений $\delta$ и $\gamma$ вариантов сильно сокращается, что улучшает вычислительную сложность используемых для решения этой задачи алгоритмов. 

Задачу изоморфизма помеченных графов также можно поставить более общем и полезном на практике виде: найти такой частичный подграф графа $G$, который будет изоморфен в обозначенном выше смысле графу $G'$. Такую задачу называют задачей поиска паттерна на помеченном графе.

В статье \cite{patmat} приведён алгоритм нахождения граф-паттерна на помеченном графе в случае, если дуги графа не помечены (то есть в формулировке нет множества $M$), а метки вершин графа $G'$ -- уникальны. В подготовленной мной выпускной бакалаврской работе приведено описание этого алгоритма, а также его модификация для случая, когда метки графа $G'$ не уникальны, и допускаются метки на дугах. Такая модификация расширяет применение рассмотренных в статье \cite{patmat} методов на целый класс прикладных задач. Например, при поиске паттернов на фотографиях часто бывает так, что разыскиваемый паттерн состоит из нескольких одинаковых частей. Таков, например, паттерн лица или солнцезащитных очков. Более того, на таких паттернах зачастую важно учитывать расстояние между их частями, которое отлично моделируется весами на дугах графов-паттернов. В данной работе показано, как учитывать веса на дугах графов-паттернов для нахождения соответствующего данному паттерну частичного подграфа.

Кроме распознавания образов на фотографиях, поиск паттернов на помеченных графах может использоваться и как инструмент социальной инженерии. Так, например, в 2016 году в средствах массовой информации широко освещалась деятельность так называемых групп смерти (\cite{sinkit}), получивших распространение в различных социальных сетях. При наличии полученной административными методами переписки жителей определённого города или района, можно построить граф, вершинами которого будут аккаунты пользователей, а дугами будут связаны те пользователи, которые вели друг с другом переписку. Особым образом следует пометить вершины, соответствующие аккаунтам <<целевой аудитории>> групп смерти -- подростков в возрасте от 13 до 19 лет, а также дуги, соответствующие переписке с различными стоп-словами типа <<суицид>>, <<4:20>> и пр. После этого следует составить граф-паттерн, выглядящий, например, как граф-звезда (т.е. связный граф, степень всех вершин которого, кроме одной, равна 1 \cite{stargraph}) с 10 листьями-подростками, а дуги графа-паттерна должны быть помечены в обозначенном выше смысле. После того, как паттерн построен, можно применить алгоритм из данной работы и найти все соответствующие ему подграфы, после чего, проводя их анализ, установить координаторов <<групп смерти>>.

Все представленные в данной работе алгоритмы были реализованы на языке Python с использованием библиотеки GraphTool. Оба этих проекта -- проекты с открытым исходным кодом и исключительно полной документацией (\cite{pythdoc}, \cite{graphtool}), которая достаточно активно использовалась в процессе разработки. Для контроля версий и работы за разными машинами также использовалась система контроля версий Git, исходный код которой также открыт. Несмотря на обилие открытых источников, для более подробного ознакомления с этой системой мною была выбрана книга \cite{gitbook}.