\section{Введение}

Данная работа посвящена решению одной интересной задачи, возникающей на стыке двух разделов теории графов: нестандартной достижимости и теории случайных графов. Коротко раскроем предмет исследования каждого из разделов, познакомим читателя с основными имеющимися результатами, а затем перейдём к описанию исследуемой задачи.

Задачи на графах с нестандартной достижимостью появляются там, где на прохождение по дугам графа накладываются некоторые ограничения. Пусть, например, стоит задача найти оптимальный по времени путь из Ростова в Москву на автомобиле. Для её решения обычно строят граф дорожной сети, в котором вершинами являются населённые пункты, а рёбрами -- соединяющие их дороги, каждому ребру присваивается вес -- длина дороги, после чего задача решается одним из известных алгоритмов поиска кратчайшего пути. Однако такое решение рассчитано лишь на случай, когда автомобиль может двигаться по каждой из дорог со своей максимальной скоростью. Возникающие на дорогах пробки могут нарушить это условие, увеличивая вес рёбер, на которых они возникают, в определённые временные интервалы. Задачи подобного рода называются задачами на временную достижимость. Эту же задачу можно сформулировать, например, для гипотетического автомобиля на солнечных батареях, который может беспрепятственно двигаться лишь днём и под открытым небом, а в туннелях и ночью -- непродолжительное время. Если сформулировать эту задачу более строго, получим другой тип достижимости, для которого будет применяться подход, отличный от подхода для временной достижимости. Несмотря на это, почти ко всем задачам на нестандартную достижимость уместно применять следующий общий подход (как это делается в, например, в \cite{erus}): строится новый граф с большим количеством вершин (как правило, каждой вершине исходного графа в соответствие ставятся $n$ вершин нового графа), затем определённым образом каждому ребру исходного графа в соответствие ставится несколько рёбер нового графа (это соответствие сильно зависит от типа достижимости), а затем доказывается теорема о том, что кратчайший путь в обычном смысле на построенном графе определённым образом соотносится с кратчайшем путём в нестандартном смысле на исходном графе. В отдельных случаях, однако, перестроить граф оказывается либо невозможным, либо такое перестроение сильно (нелинейно) увеличивает исходный граф, что приводит к увеличению асимптотической сложности используемых алгоритмов. В таком случае иногда оказывается удачной идея модифицировать сам алгоритм поиска кратчайших путей. Эта идея является ключевым моментом данной работы.

Любая задача на нестандартную достижимость имеет дело с весьма определённым графом, для которого всегда существует определённый кратчайший путь (или достоверно известно, что такого пути нет). Теория случайных графов же рассматривает некоторые вполне естественные распределения графов и выясняет свойства "среднего" графа такого распределения. Пожалуй, самой простой и распространённой здесь является модель Эрдеша–Реньи \cite{erdyosh}. В обобщённом варианте модели построение случайного графа происходит следующим образом. Задаётся множество $V_n = \{1, ..., n\}$, называемое множеством вершин, а неориентированное ребро, соединяющее вершины $i, j \in V_n, i < j$, появляется независимо от других рёбер с вероятностью $p_{i,j}$. Для удобства обозначим $\overline{p} = \{p_{i,j}\}_{1 \le i < j \le n}$. Все образованные таким образом неориентированные графы образуют вероятностное пространство
$$	G(n, \overline{p}) = (\Omega_n, \mathcal{F}_n, P_{n, \overline{p}}), $$где
$$\Omega_n = \{G(V_n, E)\},$$ 
$$\mathcal{F}_n = 2^{\Omega_n},$$ 
$$P_{n, \overline{p}}(G) = \prod\limits_{(i,j) \in E} p_{i, j} \cdot \prod\limits_{(i,j) \notin E} (1 - p_{i, j}).$$

Для того, чтобы выяснить, какова вероятность выполнения того или иного утверждения для случайного графа, берут подмножество $\mathcal{A} \subset \mathcal{F}_n$ множества всех случайных графов и вычисляют для него вероятность 
$$ P(\mathcal{A}) = \sum_{G \in \mathcal{A}} P_{n, \overline{p}}(G). $$

В данной работе решается задача о поиске пути на графе, достижимость на котором определяется случайно, но зависит не только от наперёд заданного закона распределения, но и от предыдущих переходов по дугам графа.