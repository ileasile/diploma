\section*{Частные случаи}
\par Пусть $S =\{b \in \mathbb{B}\ |\ q(b)\ne 0\}$ - множество допустимых конфигураций. Пусть также $\mathfrak{X} = M_{|S|}([0;1])$ - множество квадратных матриц порядка $|S|$ с элементами из $[0;1]$. Элементы матрицы $\chi \in \mathfrak{X}$ будем обозначать так: $\chi_{b_s,b_f}$, где $b_s, b_f \in S$. Теперь придадим этим матрицам некоторый смысл в рамках нашей задачи. 
\par Каждой паре вершин $s, f \in X$ поставим в соответствие матрицу $\chi(s,f) \in \mathfrak{X}$ такую, что число $\chi_{b_s,b_f}$ означает вероятность того, что начав движение из вершины $s$ в конфигурации $b_s$ можно добраться в вершину $f$ в конфигурации $b_f$. 
\par Легко видеть, что задача в первоначальной постановке сводится к задаче определения матрицы $\chi$ для пары из начальной и конечной вершин, и последующему вычислению суммы $\sigma = \sum\limits_{b\in S} \chi_{b_s,b}$, где $b_s$ - стартовая конфигурация. $\sigma$ и будет ответом на задачу.
\par Найдём теперь такую матрицу для некоторых пар вершин на некоторых специальных графах. Для простоты рассмотрим лишь частный случай общей задачи. Пусть $G(X, U, f)$ - граф с условиями как указано выше, причём $U = U_0 \cup U_R \cup U_G$,\\ $S = \left\{\begin{pmatrix}
0\\ 
1
\end{pmatrix},
\begin{pmatrix}
1\\ 
0
\end{pmatrix}
 \right\}$. 