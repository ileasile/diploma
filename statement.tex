\section{Исходная задача}

\subsection{Постановка задачи}
\par Пусть $L$ - непустое конечное множество \textbf{(множество меток)}. Пусть $G(V, E)$, $G'(V', E')$ -- неориентированные связные графы. Будем называть граф $G$ архивным графом, граф $G'$ -- графом-паттерном, или шаблонным графом.

Введём отображения $l : V \to L$, $l' : V' \to L$, сопоставляющие вершинам архивного и шаблонного графов соответствующие метки. При этом мы требуем, чтобы введённые на шаблонном графе метки были уникальны, т.е. отображение $l'$ -- инъективно.

\begin{defn}
	Совпадением на графе $G$ будем называть частичный подграф $\widehat{G}(\widehat{V}, \widehat{E})$ графа $G$ такой, что:
	\begin{enumerate}
		\item Существует биективное отображение $m_{\widehat{G}}: V' \to \widehat{V}$.
		\item $\forall v' \in V': l'(v') = l(m_{\widehat{G}}(v'))$
		\item $\forall (v^{\prime}_1, v^{\prime}_2) \in E': (m_{\widehat{G}}(v^{\prime}_1), m_{\widehat{G}}(v^{\prime}_2)) \in E$
	\end{enumerate}
\end{defn} 

Пусть для вершины $v \in V$ существует некоторое совпадение  $\widehat{G}(\widehat{V}, \widehat{E})$ на графе $G$ такое, что $v \in \widehat{V}$. Тогда вершину $v$ будем называть подходящей паттерну $G'$ по совпадению $\widehat{G}$, иначе -- неподходящей. В случае, если вершина $v$ подходит по совпадению $\widehat{G}$, вершину $m_{\widehat{G}}^{-1}(v) \in  V'$ назовём соответствующей данной вершине $v$. Заметим, что вершина $v$ может подходить паттерну по нескольким совпадениям, но в силу инъективности отображения $l'$ ей может соответствовать лишь одна вершина $v' \in  V'$.

Через $\widetilde{V} \subseteq V$ обозначим множество всех подходящих паттерну $G'$ вершин. Наша задача и будет состоять в отыскании этого подмножества. Опишем используемый нами алгоритм.

Пусть $T = \emptyset \cup \mathcal{C}_{V'}^1$ -- все 0-элементные и 1-элементные подмножества множества вершин графа-паттерна. Построим отображение $f_0 : V \to T$, заданное следующим:
\begin{equation}
v' \in f_0(v) \Leftrightarrow l(v) = l'(v').
\end{equation}

Нетрудно убедиться, что в силу инъективности отображения $l$, введённое отображение $f_0$ действительно имеет областью значений множество $T$.

Алгоритм будет строиться на изменении отображения $f_0$, поэтому для удобства нам потребуется ввести операцию над подобными отображениями. Пусть $f_1, f_2 : V \to 2^{V'}$. Обозначим $f_2 = Exclude(f_1, v_0 (\in V), v^{\prime}_0 (\in V'))$, если выполнено следующее:
\begin{enumerate}
	\item $\forall v \ne v_0 \in V: f_1(v) = f_2(v)$.
	\item $v^{\prime}_0 \notin f_2(v_0)$.
	\item $\{v^{\prime}_0\} \cup f_2(v_0) = f_1(v_0)$
\end{enumerate}

Ясно, что по отображению $f_1$ легко построить отображение $Exclude(f_1, v_0, v^{\prime}_0)$, просто исключая вершину $v^{\prime}_0$ из множества $f_1(v_0)$.

\subsection{Алгоритм исключения по локальным условиям}

Для начала дадим несколько определений.

\begin{defn}
	Пусть дан граф $G(V, E)$. Пусть $v \in V$. Тогда обозначим через $Adj(v)$ множество смежных с ней вершин: $Adj(v) = \{u \in V | \exists (v, u) \in E \}$. Если $\widetilde{V} \subset V$, то $Adj(\widetilde{V}) = \bigcup\limits_{\widetilde{v} \in \widetilde{V}} Adj(\widetilde{v})$. В частности, $Adj(\emptyset) = \emptyset$.
\end{defn} 

\begin{defn}
	Пусть $G(V, E)$, $G'(V', E')$ -- архивный и шаблонный графы соответственно. Пусть задано отображение $f : V \to T$, где множество $T$ определено выше. Путь также $v \in V$. Назовём предикатом локальных условий следующий предикат:
	\begin{equation}
		LCC(f, v) = \forall u' \in Adj(f(v)) \exists u \in Adj(v) : u' \in f(u).
	\end{equation} 
\end{defn} 

Ниже приведён алгоритм LCCE -- исключения по локальным условиям.
\begin{algorithm}
	\Large
	\KwIn{графы $G(V, E)$, $G'(V', E')$, отображение $f : V \to T$, число итераций $N$}
	\KwOut{измененённое отображение $f$}
	\Begin(LCCE){
		\For{$i = 1, 2, .., N$}{
			\For{$v \in V$}
			{
				\If{$\neg LCC(f, v)$}
				{
					$f(v) := \emptyset$
				}
			}
		}
		\Return{$f$}
	}
\end{algorithm}

